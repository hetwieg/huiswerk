\documentclass[a4paper,titlepage]{artikel1}
\usepackage[dutch]{babel}
%\usepackage{a4wide}
%\usepackage{eurofont}


\usepackage{ucs}
\usepackage[utf8x]{inputenc}
\usepackage{fullpage}
\usepackage{url}
\usepackage{eurosans} 
\usepackage{multirow}
\usepackage{graphicx}

\setlength{\parskip}{0.2cm}
\setcounter{secnumdepth}{3}


\author{Paul Sohier 0806122\\Sebastiaan Polderman 0820738}
\title{Robotica \\ Open source robot platform}


\begin{document}
\maketitle
\tableofcontents
\newpage
\section{Open source robot platform}
Voor bijna alle soorten robots welke je wilt gebruiken zijn eigenlijk al open source varianten. Er zijn van diverse initatieven om robots te ontwikelen en te gebruiken voor educactionele doeleinden. Een goed voorbeeld hiervan is de e-puck. The e-puck was ontwikkeld in Zwitserland. Naast de e-puck zijn nog diverse andere open pakketten welke specifiek ontwikkeld zijn voor software van robots.
\subsection{E-puck}
De robot is compleet open source en hierdoor eenvoudig uit te breiden naar eigen wensen en functie eisen. Om hem te uit te breiden zijn er diverse extensies aanwezig om dit naar eigen wens te doen.
De robot zelf werkt op een 30MHz CPU gebaseerd op een PIC microcontroller. Hij heeft diverse sensoren en een camera. Tevens beschikt hij over een microfoon en speaker.

Doordat de prijs laag is is de robot goed geschikt voor gebruik op scholen en universiteiten waar het budget beperkt is. Wat wel een nadeel is dat je gebonden bent aan die specifieke hardware en de specifieke programmeertaal welke de robot in is geschreven. Wanneer je dus iets anders wilt als waarvoor de robot ontwikkeld is, is de kans groot dat je het geheel helemaal moet aanpassen. In dat geval kan je dus er beter voor kiezen om met losse hardware te werken.

\subsection{Losse hardware}
Om zelf een robot te ontwikkelen zijn er diverse mogelijkheden. Een goed voorbeeld hiervan is een Adruino bord. Hierbij kan je makkelijk een eigen robot ontwikkelen om zelf een robot geheel naar eigen wens te maken. Het voordeel hierbij is dat je compleet zelf bepaald wat je maakt en geen limitaties hebt van een standaard pakket. Het nadeel hieraan is wel dat je gemiddeld genomen een hogere kennis van bijvoorbeeld electronica nodig hebt om de robot te ontwikkelen.

De prijs van losse hardware is vaak duurder gemiddeld als een kant en klare robot. Dit komt doordat de onderdelen los vaak veel duurder zijn omdat je per stuk lossen componenten koopt, terwijl wanneer je werkt met een kant en klare robot deze korting krijgt doordat er groot ingekocht wordt.
\end{document}
