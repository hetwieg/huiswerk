\documentclass[a4paper,11pt]{report}
% Hier hebben we de preamble, alle document settings moeten hier:
\usepackage{graphicx}
\usepackage{url}
\usepackage{appendix}
\usepackage[titles]{tocloft}
\usepackage[dutch]{babel}
\usepackage[xindy,style=altlistgroup]{glossaries}
\usepackage{listings}
\usepackage{makeidx}
\usepackage{fullpage}
%\usepackage{hyperref}
% Paragrafen hebben een witregel ertussen, en geen indent tab:
\setlength{\parindent}{0.0in}
\setlength{\parskip}{0.1in}
%% Woordenlijst moet aan het begin worden geinclude:
\loadglsentries{woordenlijst}
% Onderstaande is voor de dots tussen chapter title + blz. 
\makeatletter
\renewcommand*\l@chapter[2]{%
  \ifnum \c@tocdepth >\m@ne
    \addpenalty{-\@highpenalty}%
    \vskip 1.0em \@plus\p@
    \setlength\@tempdima{1.5em}%
    \begingroup
      \parindent \z@ \rightskip \@pnumwidth
      \parfillskip -\@pnumwidth
      \leavevmode \bfseries
      \advance\leftskip\@tempdima
      \hskip -\leftskip
      #1\nobreak\normalfont\leaders\hbox{$\m@th
        \mkern \@dotsep mu\hbox{.}\mkern \@dotsep
        mu$}\hfill\nobreak\hb@xt@\@pnumwidth{\hss #2}\par
      \penalty\@highpenalty
    \endgroup
  \fi}
\makeatother

% End of title + blz.
% Table of content depth van 4, dus tm paragraph
\setcounter{tocdepth}{4}
%\renewcommand{\baselinestretch}{1.5} 1.5 regelafstand 

%Pas listings aan zodat ze duidelijker zijn
\lstset{ %
  language=bash,                % choose the language of the code
  basicstyle=\footnotesize,       % the size of the fonts that are used for the code
  numbers=left,                   % where to put the line-numbers
  numberstyle=\footnotesize,      % the size of the fonts that are used for the line-numbers
  numbersep=5pt,                  % how far the line-numbers are from the code
  showspaces=false,               % show spaces adding particular underscores
  showstringspaces=false,         % underline spaces within strings
  showtabs=false,                 % show tabs within strings adding particular underscores
  frame=lr,	                % adds left and right lines
  tabsize=2,	                % sets default tabsize to 2 spaces
  captionpos=b,                   % sets the caption-position to bottom
  breaklines=true,                % sets automatic line breaking
  breakatwhitespace=false,        % sets if automatic breaks should only happen at whitespace
%  escapeinside={\%*}{*)},         % if you want to add a comment within your code
  morekeywords={*,...}            % if you want to add more keywords to the set
}
\makeindex

% Einde preamble, begin document:
\begin{document}
% Front page:
\title{
  Vakcode\\
  Datastructuren 2
}
\author{
  Sebastiaan Polderman\\
  08xxxxx\\
  \and
  Paul Sohier\\
  0806122
}
\date{\today}
\maketitle
% Abstract. Heel kort wat het is:
\begin{abstract}\centering
Samenvatting
\end{abstract}

% Nu een voorwoordje
\chapter{Voorwoord}
Hier een prachtig voorwoord. 

% De table of contents:
\tableofcontents

% Inleiding doen we ook nog in de master file:
\chapter{Inleiding}
Inleiding van het dictaat. Blablabla natuurlijk. 
% Einde inleiding

% Nu kunnen we de losse hoofdstukken gaan includen. 
% Includen gebeurt met basename, dus zonder .tex
\chapter{Casus 3}
\section{Opdracht 1}
Opstapeling wordt veroorzaakt doordat er reeds gemaakte producten moeten
wachten op een product dat van de zijkant wordt opgezet. Opstapeling zou
dus maximaal moeten vormen bij het punt met de hoogste opzettijd (de
opzetlocatie van machine 4). 

Met X zijnde de productietijd: 

In de 10 seconde opzettijd dat er gewacht moet worden, worden er
(10/X)*3 producten opgestapeld. Na het opzetten duurt het X-10 seconden
voor er een nieuw product wordt opgezet, dus zijn er 10-X seconden om de
producten door te sturen.

(10/X)*3+1 producten moeten vervoert worden in X-10 seconden. 

30/X+1<X-10

30/X+11<X

30/X+11-X<0

-30/X-11+X>0

X^{2}-11X-30>0

X>13.26

(3600/13.26) = 271 dozen per uur per machine 

Wordt de productie groter dan dit, zal het systeem vastlopen bij machine
3, omdat er op de hoofdband een opstapeling gevormd wordt die het
systeem niet kan verwerken voordat er een nieuwe doos in de wachtrij
wordt gezet om opgezet te worden. 
\section{Opdracht 2}
Met een productietijd van gemiddeld 13.5 seconden voor iedere machine
werkt het systeem optimaal (onder de voorwaarde dat alle machines
gelijke productietijden hebben). De productie in 48 uur is ongeveer
64000 producten. 
\section{Opdracht 3}
Machine 1: 8 seconden gemiddeld

Machine 2: 8.5 seconden gemiddeld

Machine 3: 9.5 seconden gemiddeld

Machine 4: 17 seconden gemiddeld

Machine 5: 16.5 seconden gemiddeld 

Het resultaat is dat er per 48 uur ongeveer 80000 producten geproduceerd
worden. Een productiestijging van 25\%! Dus ja, door de machines in
productietijd te laten variëren is het mogelijk de productie nog verder
te verhogen zonder het systeem te laten vastlopen.
\section{Opdracht 4}
Verhoging van de snelheid van de rollenbanen zou inderdaad effect
hebben. Het opstapelende effect van de langzame opzetting bij de derde
en vierde machine zou zo sneller teniet gedaan kunne worden. Bij
versnelling van  de rollenbanen kunnen er meer dozen langs een opzetpunt
verplaatst worden voordat er gewacht moet worden voor een product dat
van de zijkant instroomt.

% Include van de biblio file, ook in de toc:
\addcontentsline{toc}{chapter}{\numberline{}Bibliografie}
\begin{thebibliography}{99}

\bibitem{bib.wikipedia}
Wikipedia, \textsl{De vrije encyclopedie}
\\\mbox{}\hfill\url{http://nl.wikipedia.org/}

\end{thebibliography}

% Verklarende woordenlijst + toc:
\addcontentsline{toc}{chapter}{\numberline{}Verklarende Woordenlijst}
%\printglossaries

% Appendix:
\appendix
\chapter{Deel 1}\label{app.deel1}


\addcontentsline{toc}{chapter}{Index}
\printindex

% Einde document
\end{document}
